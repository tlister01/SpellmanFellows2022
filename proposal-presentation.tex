\documentclass{beamer}
\usetheme{Madrid}
\usepackage{tikz}
\usepackage{graphicx}
\usepackage{ragged2e}

\definecolor{Coe}{RGB}{136, 0, 17}
\definecolor{Gold}{RGB}{211, 144, 47} 

\setbeamercolor{palette primary}{bg=Coe,fg=white}
\setbeamercolor{palette secondary}{bg=Gold,fg=white}
\setbeamercolor{palette tertiary}{bg=Coe,fg=white}
\setbeamercolor{block title}{bg=Coe}
\setbeamertemplate{itemize item}{\color{Coe}$\blacktriangleright$}
\setbeamertemplate{itemize subitem}{\color{Coe}$\blacktriangleright$} 

\title[Increased Federal Unemployment Aid]{Increased Federal Unemployment Aid, Individuals Willingness to Work, and the United States Labor Shortage}
\author[Toby Lister]{Toby Lister \linebreak \linebreak Faculty Advisors: Drew Westberg, PhD. and Ryan Baranowski, PhD.}
\date{June 8th, 2022}
\centering


\begin{document}
\maketitle 
%%%%%%%%%%%%%%%%%%%%%%%%%%%%%%%%
% Research Question
%%%%%%%%%%%%%%%%%%%%%%%%%%%%%%%%

\begin{frame}{Research Question}
	\begin{block}{}
	Did the increased unemployment insurance (UI) benefits lead to a change in individuals' willingness to work, and what effect do the increased 	benefits have on creating a labor shortage in the United States?
	\end{block}
\end{frame}

%%%%%%%%%%%%%%%%%%%%%%%%%%%%%%%%
% Motivation
%%%%%%%%%%%%%%%%%%%%%%%%%%%%%%%%

\begin{frame}{Motivation}
	\begin{itemize}
	\item Unemployment reached approximately 23 million at its peak
	\item Gov responded by passing C.A.R.E.S. Act
		\begin{itemize}
		\item Increased unemployment benefits
		\item Gave stimulus checks to millions of Americans
		\end{itemize}

	\begin{figure}[H]
 	\includegraphics[width= \linewidth]{NonFarmPayroll-Total.png}
  	\caption{Percentage change from a year ago, All Employees, Total Nonfarm Payroll between May 2007 and January 2022.}
	\end{figure}
	\end{itemize}
\end{frame}

%%%%%%%%%%%%%%%%%%%%%%%%%%%%%%%%
% Lit Review
%%%%%%%%%%%%%%%%%%%%%%%%%%%%%%%%

\begin{frame}{Literature Review}
	\begin{block}{Baicker, Golden, and Katz (1998)}
		\begin{itemize}
		\item Discussed the history of the United States Unemployment Insurance (UI) program
		\end{itemize}
	\end{block}
	
	\begin{block}{}
		\begin{itemize}
		\item Created by the \hyperlink{definitions}{\color{Gold}{omnibus}} Social Security Act of 1935
		\item Intended to provide UI benefits to individuals w/ attachement to labor force (i.e. temporarily unemployed)
		\item Three common features across U.S.
			\begin{enumerate}
			\item State System
			\item Potential for \hyperlink{definitions}{\color{Gold}{Experience Rating}}
			\item Limitation on Duration of Benefits
			\end{enumerate}
		\end{itemize}
	\end{block}
\end{frame}

\begin{frame}{Literature Review- Labor Search}
	\begin{block}{Mortenson (1977)}
		\begin{itemize}
		\item Developed a more realistic model of job search
			\begin{itemize}
			\item Allowed workers to search while employed
			\item Search intensity is a choice variable
			\item Cost of search is viewed as the value of leisure given up
			\end{itemize}
		\item Common institutional features: 
			\begin{enumerate}
			\item Benefits are paid for a specified duration
			\item Workers that quit do not qualify for UI benefits
			\end{enumerate}
		\end{itemize}
	\end{block}
	
	\begin{block}{Conclusions}
		\begin{itemize}
		\item Once features of most state UI programs are taken into account, UI benefits on measured search unemployment is theoretically ambiguous.
		\item Even though theoretical conclusion has not been tested, existing empirical evidence is consistent with it.
		\end{itemize}
	\end{block}
	
%	\begin{block}{Why it matters?}
%		\begin{itemize}
%		\item
%		\end{itemize}
%	\end{block}
\end{frame}

\begin{frame}{Literature Review- Labor Search}
	\begin{block}{Katz and Meyer (1990)}
		\begin{itemize}
		\item Looked at the importance of examining the layoff-rehire process in unemployment outcomes
		\item Found that \hyperlink{definitions}{\color{Gold}{recall}} expectations have a significant impact on individuals behavior while unemployed
		\item Found that \hyperlink{definitions}{\color{Gold}{reservation wage}} decreases and job search intenisity increased over time during unemployment
		\end{itemize}
	\end{block}
	
	\begin{block}{Why it matters?}
		\begin{itemize}
		\item Gives an understanding of how workers \hyperlink{definitions}{\color{Gold}{recall}} expextations impact their behavior
		\end{itemize}
	\end{block}
\end{frame}

\begin{frame}{Literature Review- Supply and Demand (Kind of)}
	\begin{block}{Dube (2021)}
		\begin{itemize}
		\item Looked at effects of \$600 UI boost.
		\item Used a difference-in-difference approach, compared change in \hyperlink{definitions}{\color{Gold}{replacement rate}} across different states
		\item Found 98 percentage point drop in the \hyperlink{definitions}{\color{Gold}{replacement rate}} on average.
		\item Found little impact on job gains as a result of a reduction in UI benefits
		\end{itemize}
	\end{block}
	
	\begin{block}{Why it matters?}
		\begin{itemize}
		\item Does not directly look at labor supply and demand
		\item Does suggest, in this instance, that labor supply is not significantly impacted.
		%\item Does this have to do with workers' expectations on being recalled?
		\end{itemize}
	\end{block}
\end{frame}

\begin{frame}{Literature Review- Misc.}

	\begin{block}{Gruber (1994)}
		\begin{itemize}
		\item Looks at changes in consumption behavior due to legislative variations in UI benefits
		\item Estimated that a 10 percentage point increase in \hyperlink{definitions}{\color{Gold}{replacement rate}} leads to a much smaller fall in consumption of 2.7\%%, when average decrease over this period was 7\%
		\item Suggests that decrease in consumption would have been three times larger without UI benefits.
		\end{itemize}
	\begin{block}{Remaining Questions}
	Does this mean UI benefits impacted consumption rather than labor supply and demand? Did it impact Both? To what degree?
	\end{block}
	\end{block}
\end{frame}

%\begin{frame}{Literature Review- Misc.}
%	\underline{\underline{\textcolor{red}{\textbf{\LARGE{Include?}}}}}
%	\begin{block}{Meyer (1988)}
%		\begin{itemize}
%		\item Looks at effects of the level and length of UI benefits on unemployment duration.
%		\item Using a duration model, 
%		\end{itemize}
%	\end{block}
%\end{frame}

\begin{frame}{Literature Review-Takeaways}
	\begin{block}{}
		\begin{itemize}
		\item Worker expectations play a significant role in how they respond to unemployment spells. (Katz and Meyer 1990)
			\begin{itemize}
			\item Workers that expect to be \hyperlink{definitions}{\color{Gold}{recalled}} have a lower job search intensity
			\end{itemize}
		\item Labor supply is not greatly affected by changes in UI benefits (Dube 2021)
		\item Consumption is influenced by variations in UI benefits (Gruber 1994)
		\end{itemize}
	\end{block}
\end{frame}
%%%%%%%%%%%%%%%%%%%%%%%%%%%%%%%%
% Method and Data
%%%%%%%%%%%%%%%%%%%%%%%%%%%%%%%%

\begin{frame}{Methodology}

	\begin{block}{}
	The literature suggests two possible methods, although there are certainly more that were not discussed in the papers.
	\end{block}
	
	\vspace{5mm}
	
	\begin{block}{Possible Methods}
		\begin{enumerate}
		\item Labor Search Model
		\item Analysis of Change in Labor Supply and Labor Demand
		\end{enumerate}
	\end{block}
\end{frame}

\begin{frame}{Data}
	\begin{block}{Potential Data Sources}
		\begin{enumerate}
		\item Department of Labor
		\item Bureau of Economic Analysis (BEA)
		\item Federal Reserve Economic Data (FRED)
		\item Continuous Wage and Benefit Dataset (Meyer 1988)
		\item Panel Study of Income Dynamics (Gruber 1994)
		\end{enumerate}
	\end{block}
\end{frame}
%%%%%%%%%%%%%%%%%%%%%%%%%%%%%%%%
% Timeline
%%%%%%%%%%%%%%%%%%%%%%%%%%%%%%%%

\begin{frame}{Anticipated Timeline}
\begin{itemize}
\item Week 1-4
	\begin{itemize}
	\item Literature Review (see list of \hyperlink{add}{\color{Gold}{additional readings}}) and Preliminary Data Collection
	\end{itemize}
\item Week 5-8
	\begin{itemize}
	\item Model Construction and Empirical Work
	\end{itemize}
\item Week 9-10
	\begin{itemize}
	\item Make Poster and Final Presentation
	\end{itemize}
\end{itemize}	
\end{frame}

\begin{frame}{Additional Readings}
\label{add}
\fontsize{7.5pt}{10pt}\selectfont
\begin{block}{}
\begin{itemize}
\item Leduc, Sylvian and Zheng Liu.``Uncertainty Shocks Are Aggregate Demand Shocks''. \textbf{Journal of Monetary Economics}. 2016.
\item Ehrenberg, Ronald G. and Ronald L. Oaxaca.``Unemployment Insurance, Duration of Unemployment, and Subsequent Wage Gain''. \textbf{The American Economic Review}. 1976.
\item Kong, Edward and Daniel Prinz.``Disentangling policy effects using proxy data: Which shutdown policies affected unemployment during the COVID-19 pandemic?''.\textbf{Journal of Public Economics}. 2020.
\item Farrell, Diana and Peter Ganong, and Fiona Greig, and Max Liebeskind, and Pascal Noel, and Joseph Vavra.``Consumption Effects of Unemployment Insurance during the Covid-19 Pandemic''.\textbf{JP Morgan Chase and Co. Institute}. 2020.
\item Casado, Miguel, and Britta Glennon, and Julia Lane, and David McQuown, and Daniel Rich, and Bruce Weinberg.``The Aggregate Effects of Fiscal Stimulus: Evidence From the Covid-19 Unemployment Supplement''.\textbf{National Bureau of Economic Research Working Paper Series}, 2021.
\item Marinescu, Ioana, and Daphne Skandalis, and Daniel Zhao.``Job Search, Job Posting and Unemployment Insurance During the COVID-19 Crisis''.\textbf{National Bureau of Economic Research}. 2020.
\item Bartik, Alexander and Marianne Bertrand, and Feng Lin, and Jesse Rothstein, and Matthew Unrath.``Measuring the Labor Market at the Onset of the COVID-19 Crisis".\textbf{University of Chicago, Becker Friedman Institute for Economics Working Paper}. 2020.
\end{itemize}
\end{block}
\end{frame}

%%%%%%%%%%%%%%%%%%%%%%%%%%%%%%%%
% Plan for Next week
%%%%%%%%%%%%%%%%%%%%%%%%%%%%%%%%
\begin{frame}{Plan For The Next Week}

\begin{block}{}
	\begin{itemize}
	\item Begin reading through additional papers
	\vspace{2mm}
	\item Make a running list of variables of interest
	\vspace{2mm}
	\item Try to decide on a method
	\end{itemize}
\end{block}
\end{frame}

%%%%%%%%%%%%%%%%%%%%%%%%%%%%%%%%
% References
%%%%%%%%%%%%%%%%%%%%%%%%%%%%%%%%

\begin{frame}{References}
\fontsize{9pt}{10pt}\selectfont
	\begin{block}{}
		\begin{itemize}
		\item Baicker, Katherine, Claudia Golden, and Lawrence F. Katz.``A Distinctive System: Origins and Impact of U.S. Unemployment Compensation'', \textbf{University of Chicago Press}, 1998.

		\item Dube, Arindrajit.``Aggregate Employment Effects of Unemployment Benefits During Deep Downturns: Evidence from the Expiriation of the Federal Pandemic Unemployment Compensation'', \textbf{National Bureau of Economic Research Working Paper Series}, 2021.

		\item Gruber, Jonathan.``The Consumption Smoothing Benefits of Unemployment Insurance'', \textbf{National Bureau of Economic Research Working Paper Series}, 1994.

		\item Katz, Lawrence F. and Bruce D. Meyer.``Unemployment Insurance, Recall Expectations, and Unemploymemt Outcomes'', \textbf{The Quarterly Journal of Economics}, 1990.

		\item Meyer, Bruce D..``Unemployment Insurance and Unemployment Spells'', \textbf{National Bureau of Economic Research Working Paper Series}, 1988.

		\item Mortensen, Dale.``Unemployment Insurance and Job Search Decisions''. \textbf{ILR Review}. 1977.		
		\end{itemize}
	\end{block}
\end{frame}

%%%%%%%%%%%%%%%%%%%%%%%%%%%%%%%%
% Important Definitions
%%%%%%%%%%%%%%%%%%%%%%%%%%%%%%%%

\begin{frame}{Important Definitions}
\label{definitions}
\raggedright

\textbf{Experience Rating:} a tax evaluation tool used by state unemployment insurance programs that allows states to collect unemployment taxes from employers according to the amount of unemployment insurance benefits drawn by their former employees.\\
\vspace{3mm}
\textbf{Omnibus}: A proposed law that covers a number of diverse or unrelated topics\\
\vspace{3mm}
\textbf{Recall:} To return to your job after a temporary absence; a temporary lay-off\\
\vspace{3mm}
\textbf{Replacement Rate:} the percentage of an individual's annual employment income that is replaced by another source; i.e. retirement income\\
\vspace{3mm}
\textbf{Reservation Wage:} The lowest wage a worker is willing to accept in order to participate in the labor market.\\

\end{frame}



\end{document}